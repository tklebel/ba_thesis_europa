Das Demokratiedefizit der EU kann schon als Allgemeinplatz verstanden werden. Bemüheungn um eine Behebung des PRoblems setzen an vielen verschiedenen Punkten an. Habermas und Hill skizzieren in... eine Veränderung der demokratischen Strukturen, um zu einer stärkeren LEgitimation der EU zu gelangen.

Klar verbunden mit den Demokratieschen Institutionen kann aber auch die Frage nach einer europäischen Öffentlichkeit gesehen werden. Demokratie funktioniert nicht ohne Öffentlichkeit (Quelle, zb Schildberg: 32). Es braucht einen Austausch, Konsens und Dissens. Dieser Austausch findet jedoch nur rudimentär statt. Den aktuellen Zustand im Sinne empirischer Eerhebungen über den Zustand der europäischen ÖFfentlichkeit darstellen.

Verbunden mit der Frage der ÖFfentlichkeit ist auch die Frage der Identität. Wenn ich mich als Subjekt und Mitglied in einem öffentlichen Prozess beteiligen möchte, brauche ich auch eine gewisse Art von Identifikation mit dieser Öffentlichkeit (ist eine Unterstellung: prüfen/beweisen). In diesem Zusammenhang stellt sich aber die Frage: WAs ist europäische Identität? Gibt es europäische Identität? Wer fühlt sich als Europäer_in?

Empirische Untersuchungen hierzu führen wieder zurück zu den PRobleme des Eurobarometer aus Teil 1.

Insgesamt: FRage stellen, wie das Zusammenspiel von ÖFfentlichkeit und Identität ist. Wäre eine in irgendeiner Art und Weise umzusetzender eiropäischer Öffentlichkeit ein Antrieb/Katalsysator für den Aufbau einer Europäischien Identität/des Selbstverständnis' als Europäer_in, die in der Folge zu einer stärkeren "Bindung" und Identifikation mit der EU, und also einer stärkeren Bereictschaft, sich innerhalb der EU als solidarisch zu verhalten führen könnten?