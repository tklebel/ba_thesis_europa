\documentclass[a4paper, german, oneside]{scrartcl}

\usepackage[german]{babel} %ngerman

\usepackage[utf8]{inputenc} 

\usepackage[T1]{fontenc}  
\usepackage{lmodern}

\usepackage{graphicx}

\usepackage{datetime}
\newdate{date}{19}{04}{2015}
\date{\displaydate{date}}

% Fußnote für Tabellen
\usepackage{tablefootnote}
\usepackage{amsmath}

% Auflistung in Text mit \begin{inparaenum}[(i)] und dann \item
\usepackage{paralist}
% Bessere Formatierung für Tabellen
\usepackage{booktabs}

% befehl: smaller, für relative schriftgröße
\usepackage{relsize}

% links
\usepackage{hyperref}


\usepackage[babel,german=quotes]{csquotes}  % Deutsche Anführungszeichen mit \enquote{}

%%%%%%%%%%%%%%%%%%%%%%%%%%%%%%%%%%%%%%%%%%%%%%%%%%%%%%%%%%%%%%%%%%%%%%%%%%%%%%%%%%%%%%%%%%%%%%%%%%%%%%%%%%%%%%%%%%%%%%%%%%%%%%%%%%%%%%%%%%%%%%%%%%%%%%%%
%%%%%%%%%%%%%%%%%%%%%%%%%%% Bibliographie %%%%%%%%%%%%%%%%%%%%%%%%%%%%%%%%%%%%%%%%%%%%%%%%%%%%%%%%%%%%%%%%%%%%%%%%%%%%%%%%%%%%%%%%%%%%%%%%%%%%%%%%
\usepackage[nottoc]{tocbibind}
\usepackage[style=authoryear, backend=biber, firstinits=false]{biblatex} % firstinits kürzt die Vornamen ab
\addbibresource{literatur_expose.bib}

\DefineBibliographyStrings{english}{andothers={et\ al\adddot}} % "u.a." zu "et al." 
\DefineBibliographyStrings{german}{andothers={et\ al\adddot}} % "u.a." zu "et al." 
% \DefineBibliographyStrings{english}{bibliography = {References}}
\DefineBibliographyStrings{german}{references = {Literaturverzeichnis}}

% Internetquellen und sonstige Quellen separat
\defbibheading{LV}{\section*{Literaturverzeichnis}}
\defbibheading{IQ}{\section*{Internetquellen}}
 
\defbibfilter{LV}{\not\keyword{Internet}}
\defbibfilter{IQ}{\keyword{Internet}}
 


%richtige Reihenfolge bei mehreren Autoren
\DeclareNameAlias{sortname}{last-first}

% %keine Klammern in Biblio
% \renewbibmacro*{date+extrayear}{%
%   \iffieldundef{year}
%     {}
%     {\printtext{\printdateextra}}}

% \renewcommand*{\mkbibnamefirst}[1]{#1\addcomma} % #1 


% Buchkapitel mit seperatem Datum
\DeclareLabeldate{%
\field{origdate}
\field{date}
\field{eventdate}
\field{urldate}
\literal{nodate}
}

\renewbibmacro*{publisher+location+date}{%
\printlist{location}%
\iflistundef{publisher}
{\setunit*{\addcomma\space}}
{\setunit*{\addcolon\space}}%
\printlist{publisher}%
\setunit*{\addspace}%
\iffieldsequal{labelyear}{year}{%
\usebibmacro{date}}{\printdate}
\newunit}


%%%%%%%%%%%%%%%%%%%%%%%%%%%%%%%%%%%%%%%%%%%%%%%%%%%%%%%%%%%%%%%%%%%%%%%%%%%%%%%%%%%%%%%%%%%%%%%%%%%%%%%%%%%%%%%%%%%%%%%%%%%%%%%%%%%%%%%%%%%%%%%%%%%%%%%%%
%%%%%%%%%%%%%%%%%%%%%%%%%%%%%%%%%%%%%%%%%%%%%%%%%%%%%%%%%%%%%%%%%%%%%%%%%%%%%%%%%%%%%%%%%%%%%%%%%%%%%%%%%%%%%%%%%%%%%%%%%%%%%%%%%%%%%%%%%%%%%%%%%%%%%%%%%%



%%%%%%%%%%%%%%%%%%%%%%%%%%%%%%%%%%%%%%%%%%%%%%%%%%% Zitieren %%%%%%%%%%%%%%%%%%%%%%%%%%%%%%%%%%%%%%%%%%%%%%%%%%
% Commands für Textcite und Parencite mit Dropdownmenü
\newcommand{\citet}[1]{\textcite{#1}} 
\newcommand{\citep}[1]{\parencite{#1}}

% % Blockzitat mit Anführungszeichen
% \renewcommand*{\mkblockquote}[4]{\enquote{#1}#2#4#3}

% Blockzitat etwas kleiner
\newenvironment*{smallquote}
  {\quote\smaller}
  {\endquote}

% Now we instruct csquotes to use the new environment:

\SetBlockEnvironment{smallquote}
%%%%%%%%%%%%%%%%%%%%%%%%%%%%%%%%%%%%%%%%%%%%%%%%%%


% fix weird problem with unicode
\DeclareUnicodeCharacter{00A0}{ }


\title{Europäische Öffentlichkeit und Europäische Identität}
\subtitle{Exposé zu einer Bachelorarbeit im Kurs: \emph{Gesellschaft, Kultur, sozialer Wandel: Das Modell des Europäischen Wohlfahrtsstaates (SS 2015)}}
\date{\today}
\author{Thomas Klebel; 1073073}

%%%%%%%%%%%%%%%%%%%%%%%%%%%%%%%%%%%%%%%%%%%%%%%%%%%%%%%%%%%%%%%%%%%%%%%%%%%%%%%%%%%%%%%%%%%%%%%%%%%%%%%%%%%%%%%%%%%%%%%%%%%%%%%%%%%%%%%%%%%%%%%%%%%%%%%%%%
%%%%%%%%%%%%%%%%%%%%%%%%%%%%%%%%%%%%%%%%%%%%%%%%%%%%%%%%%%%%% Ende der  Präambel %%%%%%%%%%%%%%%%%%%%%%%%%%%%%%%%%%%%%%%%%%%%%%%%%%%%%%%%%%%%%%%%%%%%%%%%%
%%%%%%%%%%%%%%%%%%%%%%%%%%%%%%%%%%%%%%%%%%%%%%%%%%%%%%%%%%%%%%%%%%%%%%%%%%%%%%%%%%%%%%%%%%%%%%%%%%%%%%%%%%%%%%%%%%%%%%%%%%%%%%%%%%%%%%%%%%%%%%%%%%%%%%%%%%

\begin{document}
\maketitle
\tableofcontents

\section{Themenexplikation}
Das Demokratiedefizit der EU kann mittlerweile als Allgemeinplatz gesehen werden. Eine fehlende demokratische Legitimierung, eine fehlende \emph{europäische} Öffentlichkeit und auf den vermeintlichen Gegensatz zwischen Nationalstaaten und der Europäischen Union anspielende Beschwerden über Gurkenkrümmungsverordnungen prägen das Bild. Bemühungen um eine Veränderung des status quo setzen an vielen verschiedenen Punkten an. \citet{grozelier_democracy_2013_habermas} und \citet{grozelier_europes_2013_hill} skizzieren eine Veränderung der demokratischen Strukturen, um zu einer stärkeren Legitimation der EU zu gelangen.

Direkt verbunden mit der Frage nach den demokratischen Institutionen der Europäischen Union ist aber auch die Frage nach einer europäischen Öffentlichkeit. Demokratie funktioniert nicht ohne Öffentlichkeit (Quelle, zb Schildberg: 32). Es braucht einen Austausch über zur Diskussion stehende Themen, Konsens und Dissens in Bezug auf europäische Entscheidungen. Dieser Austausch findet jedoch nur rudimentär statt. In der geplanten Arbeit soll eine Bestandsaufnahme über den aktuellen Zustand der europäischen Öffentlichkeit versucht werden. Ich werde der Frage nachgehen, welche Arten von (politischer) Öffentlichkeit es in der soziologischen Theorie gibt, und welche Art der Öffentlichkeit am ehesten in der Lage scheint, eine \emph{europäische Öffentlichkeit} zu bilden.

Für diese Fragestellung werde ich mich an den von \citet{schildberg_politische_2010} vorgebrachten Positionen in der Forschung orientieren: \blockquote[{\cite[32]{schildberg_politische_2010}}]{Während \emph{erstens}, die mangelnde institutionelle Demokratisierung der EU als Ursache für das Öffentlichkeitsdefizit veranschlagt wird und damit einhergehend institutionelle Reformen als notwendig angesehen werden, wird dem \emph{zweitens} entgegengesetzt, dass für die Demokratisierung der EU zunächst eine Öffentlichkeit gegeben sein muüsste, damit solche Reformen überhaupt wirken könnten. \emph{Drittens} wird das Fehlen einer europäischen Öffentlichkeit auf die Sprachenvielfalt innerhalb der EU zurückgeführt oder mit dem Fehlen europäischer Massenmedien erklärt.}

Verbunden mit der Frage der Öffentlichkeit ist auch die Frage der Identität. Für die Beteiligung an einem öffentlichen Prozess der Auseinandersetzung über europäische Themen scheint eine gewisse Identifikation mit \emph{Europa} im Allgemeinen, und der \emph{Europäischen Union} im Speziellen eine gewisse Voraussetzung zu sein. Insofern scheinen Öffentlichkeit und Identität in einem Teufelskreis gefangen zu sein: Als Basis eines (kulturellen) Identitätsempfindens können einerseits nach \textcite[28]{segers_konstruktion_1999} \enquote{stabile Traditionen und ein generationsübergreifendes, legitimiertes Geschichtsbewußtsein}, andererseits, und im Gegensatz dazu, Kommunikationszusammenhänge fungieren:  \blockquote[{\cite[160]{eder_integration_1999}}]{Wenn ein Gemeinsames identifiziert wird, dann nicht auf Grund von Geschichte, sondern deswegen, weil sich Institutionen der Kommunikation kultureller Sinnzusammenhänge ausgebildet haben, die festlegen, was als Gemeinsames kommuniziert werden soll.} Wenn aber einerseits kein geteiltes Geschichtsbewußtsein vorhanden ist, und andererseits für die Bildung einer Öffentlichkeit eine geteilte Identität vonnöten ist, die nur durch Kommunikation über Gemeinsames entstehen kann, so scheint aus diesem Kreis kein Ausweg möglich.

Neben der theoretischen Frage nach den Voraussetzungen, die zur Herausbildung einer europäischen Identität führen könnten, lassen sich auch die empirische Frage danach stellen, ob, und wenn ja, welche Personengruppen sich heute als Euroäer\_innen fühlen. Diese Frage führt insofern zurück zur Frage der Öffentlichkeit, als das gängige Instrument für die Messung dieses Sachverhalt, das \emph{Eurobarometer}, ursprünglich als Instrument geschaffe wurde, um eine europäische Öffentlichkeit zu konstruieren (und diese konstruierte Meinung dann in der Folge zu messen).
 (Vgl. \textcite[542]{weichbold_eurobarometer_2009})


Als weitere interessante Fragestellung erscheint die Bedeutung sozialer Schichten auf den Grad der Identifikation mit der EU, und in der Folge auf den Grad der Partizipation an demokratischen Institutionen der EU (Wahlen, aber auch die Beteiligung an einer wie auch immer gearteten \enquote{europäischen Öffentlichkeit}). Dieser Aspekt muss aber in der geplanten Arbeit unbeachtet bleiben, soll der Rahmen der Arbeit nicht komplett gesprengt werden.


Die zentrale Fragestellung ist also: \emph{In welchem Verhältnis stehen europäische Öffentlichkeit und europäische Identität zueinander?} Anknüpfend daran die hypothetische Frage: \emph{Wäre es möglich, durch eine Stärkung der europäischen Öffentlichkeit mehr Identifikation mit der EU und damit mehr Legitimation der EU selbst zu erreichen?}

Insgesamt: FRage stellen, wie das Zusammenspiel von ÖFfentlichkeit und Identität ist. Wäre eine in irgendeiner Art und Weise umzusetzender eiropäischer Öffentlichkeit ein Antrieb/Katalsysator für den Aufbau einer Europäischien Identität/des Selbstverständnis' als Europäer\_in, die in der Folge zu einer stärkeren "Bindung" und Identifikation mit der EU, und also einer stärkeren Bereictschaft, sich innerhalb der EU als solidarisch zu verhalten führen könnten?



\section{Fragestellungen}

Deliberative Demokratie: Gibt es Partizipation? Wer Partizipiert? 

Öffentlichkeit: Gibt es eine solche? Wie könnte sie aussehen? Wie sieht es mit Massenmedien in der EU aus?
Guter Punkt bei der ÖFfentlichkeit: Rezeption des Eurobaremoters (der ja eine europ Meinung abbilden soll) wieder nur aus nationalsstaatlicher PErspektive.

Identität
Formen von Identität?
Was könnte eine europ Identität darstellen?
empirische Befunde: FÜhlt sich jemand als Europäer\_in?
Frage nach der Schicht, bezogen auf junge Menshcen: Wie steht es mit der IDentität bei der "Oberschicht". Wie bei der "Unterschicht?"



\section{Vorläufige Gliederung}

\section{Zeitplan}

\printbibliography[filter=LV]
% \printbibliography[heading=IQ, filter=IQ]
\end{document}